\section{Introduction and Motivations}
Several groups in this course have simulated roundabouts and crossroads before. Our work is a development of and in addition to �Traffic Dynamics�, written by Tony Wood and Bastian B�cheler in May 2010 \cite{buechlerwood}. In difference to their simulation we added pedestrians and implemented crossroads with lights instead of �priority to the right� organisation. They showed impressively, that roundabouts are much more efficient than crossroads, nearly independent of the car density. They have concluded, that �their model confirms, that the increase in popularity of roundabouts over the last years is justified�. In our view one important parameter was missing: the pedestrian density. 
As we have lived so far in cities, we have had occasions enough to observe that in the mornings and evenings some large roundabouts are just blocked, when pedestrians are allowed to cross the streets, especially when in the middle of the roundabout is a station for trams or buses. Depending on the pedestrian density we have implemented three 
different signalisation modes in the crossroads. For high pedestrian densities there won't be any conflicts between pedestrians and cars. So we thought that at least at this stage, crossroads may be in advantage to roundabouts. 

